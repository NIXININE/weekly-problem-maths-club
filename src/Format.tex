\documentclass[11pt]{article}
\usepackage[top=1.0cm, bottom=1.0cm, left=1.5cm, right=1.5cm, includehead, includefoot, heightrounded]{geometry}
\usepackage{aleph-comandos}
\usepackage{latexsym}
\usepackage{amssymb}
\usepackage{bm}
\usepackage{amsmath}
\usepackage{graphicx}
\usepackage{epsfig}
\usepackage{caption}
\usepackage{blindtext}
\usepackage{multicol}
\usepackage{ragged2e}
\usepackage{tikz}
%\usepackage{moderncvstyleclassic}
%\usepackage{algorithmic}
\usepackage{fancyhdr} % To use headers
\usepackage{graphicx} %To use graphics package
\usepackage[textfont=bf, font = normalsize]{caption} %To use bold and normal size font for table and figure captions
\usepackage{floatrow} %To have table captions above the table
\usepackage{float}
\usepackage{algorithm}
\usepackage{algpseudocode}
\usepackage{framed}
%\usepackage{csquotes}
\floatstyle{plaintop}
\restylefloat{table}
%\floatstyle %To have table captions above the table
%\restylefloat{table} %To have table captions above the table
\usepackage{subfigure} % To use subfigures
\usepackage{algorithm}
\floatsetup{font = normalsize} %Size of font in
\renewcommand{\headrulewidth}{0pt}
%\newcommand{\notimplies}{\;\not\!\!\!\implies}
%\newtheorem{theorem}{Theorem}[chapter]
%\newtheorem{proof}{Proof}[chapter]
%\newtheorem{corollary}{Corollary}[chapter]
%\newtheorem{lemma}[theorem]{Lemma}
%\newtheorem{preposition}{Preposition}[chapter]
%\newtheorem{definition}{Definition}[chapter]
\newtheorem{remarks}{Remark}
\newtheorem{note}{Note}
%\newtheorem{example}{Example}[chapter]
%\newtheorem{question}{Question}[chapter]
%\newtheorem{solution}{Solution}[chapter]
%\newtheorem{exercise}{Exercise}[chapter]
\newtheorem{learning}{Self-Learning}
\newtheorem{teaser}{Brain Teaser}
\newcommand*\circled[1]{\tikz[baseline=(char.base)]{
		\node[shape=circle,draw,inner sep=2pt] (char) {#1};}}
 
%\floatstyle %To have table captions above the table
%\restylefloat{table} %To have table captions above the table
\usepackage{subfigure} % To use subfigures
\floatsetup{font = normalsize} %Size of font in tables

\begin{document}

\begin{tabular}{cl}
	\begin{tabular}{c}
			\includegraphics[width=2.28cm, height=2.2cm]{./Logos/IIITR_logo_only}	
	\end{tabular}
	
	\begin{tabular}{l}
		\parbox{0.75\linewidth}{
			\begin{center}
				
				{\large \bf Indian Institute of Information Technology Raichur, Karnataka} \\
				\vskip 2mm
				{ \textbf{\color{blue} \Large MATHEMATICS CLUB}} \\
				\vskip 2mm
				%\hrule %\vspace{.05cm}
			\end{center}}
		\end{tabular} 
	\end{tabular}
	\hrule
	\begin{center}
		{\bf Weekly Problem - WP2501 \\ Date: \today \\ Last Date to Submit the Solution is November 01, 2025}
	\end{center}
\hrule

%\begin{center}
%\textbf{Indian Institute of Information Technology Raichur}\\[1em]
%\textbf{Mathematics Club}\\[1em]
%\textbf{WP2501}\\[1em]
%\textbf{April 2024}\\[1em]
%\textbf{Dr. Bharat Soni}\\[2em]
%
%\textbf{\Large Weekly Problem}
%\end{center}

\vspace{0.5 cm}
\begin{framed}
\section*{Exercise}
Consider a $4 \times 4$ matrix whose entries are the integers $1$ through $16$
written in order left to right, top to bottom. Choose any set of four entries with
the rule that exactly one chosen entry lies in each row and exactly one chosen
entry lies in each column (equivalently: pick one entry from every row and the
chosen column indices are all distinct). Prove that the sum of the four chosen
entries is always $34$. \textbf{Henceforth, generalize the result for an $N \times N$ matrix}.

\medskip
\noindent \textbf{Clarification for students:}
``Exactly one chosen entry in each row and each column'' means no two chosen numbers share the same row or the same column.
Such a choice is called a \emph{transversal} (or a \emph{permutation selection}) of the matrix.

\medskip
\noindent \textbf{Short illustrative examples:} For the $4 \times 4$ matrix,
\[
\begin{bmatrix}
\circled{1} & 2 & 3 & 4 \\
5 & 6 & \circled{7} & 8 \\
9 & \circled{10} & 11 & 12 \\
13 & 14 & 15 & \circled{16}
\end{bmatrix}
\]
The circled numbers form one of the transversal selections. The sum of these entries $1+7+10+16=34$.
%
%\noindent Another transversal: Choose entries at positions $(1, 4), (2, 1), (3, 2)$ and $(4, 3)$, giving
%$4 + 5 + 10 + 15 = 34.$

\noindent The different choices give the same total --- you must prove why this always happens.

\end{framed}

\bigskip

\begin{center}
		\includegraphics[width=4.28cm, height=4.2cm]{solution-qr-code.png}	\\
		\noindent (Scan this to submit solution!) \\
		
		\vspace{2 cm}
		
		\textbf{\color{magenta} \Large HAPPY DIWALI}
\end{center}
\pagenumbering{gobble}
\end{document}
