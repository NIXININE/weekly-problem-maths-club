
\header

\section*{\Huge Weekly Problem}



%%%%%%%%%%%%%%%%%%%%%%%%%%%%%%%%%%%%%%%%%%%%%%%%%%%%
%% Exercise 1
%%%%%%%%%%%%%%%%%%%%%%%%%%%%%%%%%%%%%%%%%%%%%%%%%%%%

\large
\begin{exer*}

Let $a$ and $b$ be positive integers such that
\[
\frac{a^2 + b^2}{ab + 1} = k
\]
is an integer. Prove that $k$ must be a perfect square.
\vspace{0.5cm}

{\Large\textbf{Hints:}}
\begin{itemize}
    \item  First, rewrite the equation as a quadratic in $a$: 
    \[ a^2 - kba + (b^2 - k) = 0 \]
    \item If $a$ is a root, let $a'$ be the other root. Note that $a + a' = kb$ and $a \cdot a' = b^2 - k$.
		\item If $a$ is one integer root of this quadratic, what can you say about the other root?
		\item Try choosing a solution with the smallest possible $a + b$, and see what that forces.
    % \item \textbf{Infinite Descent:} Assume $k$ is not a square. Choose the solution $(a, b)$ with the \textit{smallest sum} $a+b$. Show that the second root $a'$ creates a valid solution $(a', b)$ with an even smaller sum, leading to a contradiction.
\end{itemize}

\vspace{0.5cm}
{\Large{\textbf{Examples:}}}
\begin{itemize}
    \item For $a=1, b=2$: $\frac{1^2+2^2}{1(2)+1} = \frac{5}{3} \notin \mathbb{Z}$.
    \item For $a=2, b=2$: $\frac{2^2+2^2}{2(2)+1} = \frac{8}{5} \notin \mathbb{Z}$.
\end{itemize}

The problem is asking you to understand why, whenever it is an integer, the quotient is forced to be a square — no matter what $a$ and $b$ are.


% Let $a$ and $b$ be positive integers such that:
% \[
% \frac{a^2 + b^2}{ab + 1}
% \]
% is an integer. Prove that this integer is always a perfect square.

% \vspace{0.5cm}
% \noindent \textit{(Equivalently: If $ab + 1$ divides $a^2 + b^2$, then $\frac{a^2 + b^2}{ab+1}$ must be the square of an integer.)}

% \section*{Gentle Hints}
% \textit{(Read one at a time)}

% \begin{itemize}
%     \item First, assume $a^2 + b^2 = k(ab + 1)$ for some integer $k$.
%     \item Rearrange this equation and view it as a quadratic in $a$. (You'll get something like: $a^2 - kba + (b^2 - k) = 0$.)
%     \item If $a$ is one integer root of this quadratic, what can you say about the other root?
%     \item Try choosing a solution with the \textbf{smallest possible sum} $a + b$, and see what that forces.
% \end{itemize}

% \section*{Small Example (to build intuition)}

% Take $a = 1, b = 2$, then:
% \begin{align*}
% a^2 + b^2 &= 1 + 4 = 5 \\
% ab + 1 &= 2 + 1 = 3
% \end{align*}
% Here $3$ does not divide $5$, so this pair doesn't work.

% \vspace{0.3cm}
% \noindent Now try $a = 2, b = 2$:
% \begin{align*}
% a^2 + b^2 &= 4 + 4 = 8 \\
% ab + 1 &= 4 + 1 = 5
% \end{align*}
% Again, not divisible.

% \vspace{0.3cm}
% \noindent The problem is asking you to understand why, whenever divisibility \textit{does} happen, the quotient is forced to be a square — no matter what $a$ and $b$ are.





% Let $f : [0, 2] \to \mathbb{R}$ be a continuous function satisfying the following conditions:
% \begin{align*}
%     f(0) &= 0\\
%     f(1) &= 1\\
%     f(2) &= 0\\
% 		f(x) &\geq 0 \quad \text{for all } x \in [0, 2]
% \end{align*}

% Assume that the derivative $f'(x)$ exists except at finitely many points. Let the function $I(f)$ be defined as:
% \[
% I(f) = \int_{0}^{2} \sqrt{1 + \left(f'(x)\right)^2} \, dx
% \]

% \begin{enumerate}
%     \item[\textbf{(A)}] Find the minimum possible value of $I(f)$ over all such functions $f$.
%     \item[\textbf{(B)}] Determine all functions $f$ for which this minimum is achieved.
%     \item[\textbf{(C)}] Give a geometric explanation for why your answer is the minimum.
% \end{enumerate}


% $f : [0,2] → R$ be a continuous function satisfying:\\
% % \[
% \begin{align*}
% 	f(0)&=0\\
% 	f(1)&=1\\
% 	f(2)&=0\\
% 	f(x)&=0 \hspace\forall\hspace x in [0,2]\\
% \end{align*}
% % \]
% $f'(x)$ exists except at finitely many points\\

% Define
% $I(f) = \int _0^2 \sqrt(1 + (f'(x))^2) dx$

% \begin{enumerate}
% \item[(a)] Find the minimum possible value of $I(f)$ over all such functions $f$.

% \item[(b)] Determine all functions $f$ for which this minimum is achieved.

% \item[(c)] Give a geometric explanation for why your answer is the minimum.

	
% \end{enumerate}

% Let $k$ be a positive integer.

% \section*{Part (A) — Key counting lemma}

% Consider sequences of length $2k+1$ made up of $k+1$ symbols `+1' and $k$ symbols `-1'. 
% A sequence is called \textbf{positive-prefix} if every nonempty prefix has a strictly positive sum (in the usual integer sense).

% Prove that the number of positive-prefix sequences equals
% \[ \frac{1}{2k+1} \binom{2k+1}{k} \]
% (where $\binom{n}{k}$ is the binomial coefficient).

% \paragraph{Hint:} You may use the ballot theorem or the cycle lemma; a bijective or cyclic-rotation argument also works.

% \subsection*{A concrete example:}
% For $k = 3$, the length of the sequence is $2(3)+1 = 7$. The sequence must contain four `+1's and three `-1's. The formula gives $\frac{1}{7} \binom{7}{3} = \frac{35}{7} = 5$ valid sequences.

% The 5 valid sequences and their partial sums are:
% \begin{enumerate}
%     \item \texttt{+ + + + - - -} \\
%     Sums: 1, 2, 3, 4, 3, 2, 1   
    
%     \item \texttt{+ + + - + - -} \\
%     Sums: 1, 2, 3, 2, 3, 2, 1   
    
%     \item \texttt{+ + + - - + -} \\
%     Sums: 1, 2, 3, 2, 1, 2, 1   
    
%     \item \texttt{+ + - + + - -} \\
%     Sums: 1, 2, 1, 2, 3, 2, 1   
    
%     \item \texttt{+ + - + - + -} \\
%     Sums: 1, 2, 1, 2, 1, 2, 1 
% \end{enumerate}

% \newpage

% \section*{Part (B) — The main problem}

% The integers
% \[ 1, 2, 3, \dots, 3k+1 \]
% are written down in a random order (each permutation being equally likely). 
% What is the probability that at no time during this process is the sum of the numbers written so far a positive integer divisible by 3?

% (The answer is required in closed form — factorials are permitted.)

% \paragraph{Remark to the solver:} Part (A) counts the admissible patterns of residues 1 and 2 (viewed as +1 and -1) after you remove the multiples of 3. Once you have that count, place the $k$ zeros (multiples of 3) among the safe slots and multiply by the internal permutations of numbers with the same residue to finish Part (B).


% Let a linear Diophantine equation be given by
% \[
% a_1x_1 + a_2x_2 + \dots + a_kx_k = n
% \]
% where the variables and coefficients are defined as:
% \begin{align*}
%     k, n &\in \mathbb{N} \quad \text{(natural numbers)} \\
%     x_i &\in \mathbb{N}_0 \quad \text{(non-negative integers)} \\
%     a_i &\in \mathbb{N} \quad \text{(natural numbers)}
% \end{align*}

% \begin{enumerate}
%     \item[a)] Find the total number of non-negative solutions $(x_1, x_2, \dots, x_k)$ for\\ $k \in \{1, 2, \dots, 2025\}$ and $n=2025$, given that $a_i = 1$ for all $i$.

%     \item[b)] Find the number of non-negative integers $n \le 2026$ such that the equation has an even number of non-negative solutions, given $k=4$ and coefficients $a_1=1, a_2=2, a_3=3, a_4=5$.
% \end{enumerate}


% Let a linear diophantine equation be
% \[a_1x_1+a_2x_2+\ldots+a_kx_k=n\]
% where $k,n  \in \N,
% 			 x_i  \in \N,
% 			 a_i  \in \N$

% a) Find the total number of non-negative solutions $(x_1,x_2,\ldots,x_k)$ for $k \in [1,2025]$ and $n=2025$ $ \forall a_i=1$\\
% b) Find the number of non-negative integers $n<=2026$ such that the equation has even number of non-negative solutions for $k=4$ and $a_1=1; a_2=2; a_3=3; a_4=5$.


% In a convex quadrilateral ABCD :\\

% \vspace*{-5mm}
% \begin{multicols}{2}

% \begin{tikzpicture}[scale=1.33]


% \tkzDefPoints{0/0/D,0/0/E,0/0/P2,0/0/O,0/0/A}
% \tkzDefPoint(0,0){B}
% \tkzDefPoint(3,0){C}

% % Define B by rotating D around C by 100 degrees.
% \tkzDefPointBy[rotation=center C angle -100](B) \tkzGetPoint{D}

% % STEP 2: Find the location of point A using a compatible method.
% % We define the two lines whose intersection is A.

% % --- Define Line 1 (passing through C at a 25-degree angle to CD) ---
% % We create a helper point 'P1' by rotating D around C. This is a reliable method.
% \tkzDefPointBy[rotation=center C angle 25](D)  \tkzGetPoint{P1}

% % --- Define Line 2 (passing through B at the correct angle) ---
% % The absolute angle of the line segment BA should be 210 degrees.
% % We create a helper point 'P2' on this line in a robust two-step way:
% % 1. Define a point 'P_helper' at a 210-degree angle from the origin.
% % 2. Create P2 by moving 'P_helper' by the same vector that goes from C to B.
% \tkzDefPointBy[rotation=center B angle 30](D)
% \tkzGetPoint{P2}

% % --- Find A ---
% % A is the intersection of the line (C, P1) and the line (B, P2).
% \tkzInterLL(C,P1)(B,P2)
% \tkzGetPoint{A}

% % STEP 3: Draw the quadrilateral and its diagonals.
% \tkzDrawPolygon[thick, fill=gray!10](A,B,C,D)
% \tkzDrawSegment(A,C)
% \tkzDrawSegment(B,D)

% % STEP 4: Label the vertices.
% \tkzLabelPoint[above](A){\large{A}}
% \tkzLabelPoint[below](B){\large B}
% \tkzLabelPoint[below right](C){\large C}
% \tkzLabelPoint[right](D){\large D}

% % STEP 5: Mark the given angles and equal sides.
% \tkzMarkAngle[size=0.8, fill=blue!20,](D,B,A)
% \tkzLabelAngle[pos=-1.3](A,B,D){$30^\circ$}
% \tkzMarkAngle[size=0.8, fill=green!20](A,C,B)
% \tkzLabelAngle[pos=-1.15](B,C,A){$75^\circ$}
% \tkzMarkAngle[size=0.6, fill=red!20](D,C,A)
% \tkzLabelAngle[pos=-1.3](A,C,D){$25^\circ$}
% \tkzMarkSegment[mark=||, size=3pt](C,B)
% \tkzMarkSegment[mark=||, size=3pt](C,D)

% % STEP 6: Draw the circumcircle of triangle DAC.
% \tkzDefCircle[circum](D,A,C)
% \tkzGetPoint{O}
% \tkzDrawCircle[color=orange, thick](O,A)

% % % STEP 7: Extend line CB to find E on the circle.
% \tkzInterLC(C,B)(O,A) \tkzGetPoints{E}{C_on}
% \tkzDrawLine[color=gray, dashed](B,E)
% \tkzDrawPoint[color=red, size=5pt, fill=red](E)
% \tkzLabelPoint[above right, color=red](E){\large E}

% % % STEP 8: Highlight the segments BD and CE for the proof.
% \tkzDrawSegment[color=purple, ultra thick, opacity=0.8](B,D)
% \tkzDrawSegment[color=teal, ultra thick, opacity=0.8](C,E)

% \end{tikzpicture}

% the figure satisfies:
% \begin{eqnarray}
% 	\angle ABD &=& 30^\circ \nonumber\\
% \angle BCA &=& 75^\circ \nonumber\\
% \angle ACD &=& 25^\circ \nonumber\\
% \overline{CB} &=& \overline{CD} \nonumber
% \end{eqnarray}

% Extend the line $\overline{CB}$ beyond $B$ to meet the circumcircle of triangle ${DAC}$ again at $E$
% (so $E$ is the second intersection of line $\overline{CB}$ with the circumcircle of $\triangle DAC$).\\
% \begin{center}
% \textbf{Prove that: } $\overline{CE} = \overline{BD}$
	
% \end{center}
% \end{multicols}

% \textbf{Tiny Hint:}
% \textit{Place a point $F$ on $\overline{AB}$ (or its extension) such that $\overline{CF} = \overline{CB}$.
% Show that this creates a $60^{\circ}$ angle at $C$, then relate $\overline{CE}$ and $\overline{BD}$ using the circle.
% }


% Consider a $4 \times 4$ matrix whose entries are the integers 1 through 16 written in order left to right, top to bottom.
% Choose any set of four entries with the rule that exactly one chosen entry lies in each row and exactly one chosen entry lies in each column (equivalently: pick one entry from every row and the chosen column indices are all distinct).
% Prove that the sum of the four chosen entries is always 34. Henceforth, generalize the result for an $N \times N$ matrix.
% \\

% Clarification for students: “\textit{Exactly one chosen entry in each row and each column}” means no two chosen numbers share the same row or the same column. Such a choice is called a \textbf{transversal} (or a permutation selection) of the matrix.
% \\

% Short illustrative examples: For the $4 \times 4$ matrix,\\
% \[\begin{bmatrix}
% 1 & 2 & 3 & 4 \\
% 5 & 6 & 7 & 8 \\
% 9 & 10 & 11 & 12 \\
% 13 & 14 & 15 & 16
% \end{bmatrix}\] \\

% Main diagonal selection:  $1 + 6 + 11 + 16 = 34$.\\
% Another transversal: Choose entries at positions $(1, 4), (2, 1), (3, 2) and (4, 3)$, giving  $4 + 5 + 10 + 15 = 34$.\\
% These different choices give the same total — you must prove why this always happens.
\end{exer*}

% \vfill
% \vfill
% \begin{figure}[H]
% \centering
%     \includegraphics[width=0.3\textwidth]{./solution-qr-code.png}
%     \captionsetup{labelformat = empty}
%     \caption{\Large{(Scan this to submit solution!)}}
% \end{figure}


\bigskip
\vspace{1.6cm}


\begin{multicols}{2}
		\begin{center}
		\includegraphics[width=4.28cm, height=4.2cm]{solution-qr-code.png}	\\
		\noindent (Scan this to submit solution!) \\
			
		\end{center}
\columnbreak
\Large
\textbf{\color{red}{{Top Submissions for last Week's Problem: }		}}
		
% \begin{enumerate}
% 	\item \textbf{Hamza Nazim Siddiqui}\\AD24B1023
% 	% \item \textbf{Lanka Sree Lalith Karthik}\\CS23B1042
% \end{enumerate}
\end{multicols}

\pagenumbering{gobble}


