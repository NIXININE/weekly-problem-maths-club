\documentclass[compacto,10pt,comentarios]{aleph-notas}

% It is recommended to read the documentation for this class at https://github.com/alephsub0/LaTeX_aleph-notas

% -- Additional packages
\usepackage{enumitem}
\usepackage{aleph-comandos}
\usepackage{amssymb,amsmath}
\usepackage{graphicx}
\usepackage{caption}
\usepackage{tikz}
\usepackage{tkz-euclide}
\usepackage{multicol}
\setlength{\columnsep}{7mm}

\newcommand*\circled[1]{\tikz[baseline=(char.base)]{
		\node[shape=circle,draw,inner sep=2pt] (char) {#1};}}

% -- Template details
\institution{Indian Institute of Information Technology Raichur}
\subject{Mathematics Club}
\topic{Problem Number: WP2502}
% \remark{Last Date to Submit the Solution is November 01, 2025}
\autor{Due Date: November 15, 2025 }
\dates{Date: \today}
\fonts{montserrat}

%% --> Logos
\logouno[3.5cm]{Logos/logo-blue-desktop-iiitr.png}
\definecolor{colordef}{cmyk}{0.81,0.62,0.00,0.22}

\begin{document}



\header

\section*{\Huge Weekly Problem}



%%%%%%%%%%%%%%%%%%%%%%%%%%%%%%%%%%%%%%%%%%%%%%%%%%%%
%% Exercise 1
%%%%%%%%%%%%%%%%%%%%%%%%%%%%%%%%%%%%%%%%%%%%%%%%%%%%

\large
\begin{exer*}


Let a linear diophantine equation be
\[a_1x_1+a_2x_2+\cdots+a_kx_k=n\]
where $&k,n \in \natural,
			 &x_i \in \natural,
			 &a_i \in \natural$


% In a convex quadrilateral ABCD :\\

% \vspace*{-5mm}
% \begin{multicols}{2}

% \begin{tikzpicture}[scale=1.33]


% \tkzDefPoints{0/0/D,0/0/E,0/0/P2,0/0/O,0/0/A}
% \tkzDefPoint(0,0){B}
% \tkzDefPoint(3,0){C}

% % Define B by rotating D around C by 100 degrees.
% \tkzDefPointBy[rotation=center C angle -100](B) \tkzGetPoint{D}

% % STEP 2: Find the location of point A using a compatible method.
% % We define the two lines whose intersection is A.

% % --- Define Line 1 (passing through C at a 25-degree angle to CD) ---
% % We create a helper point 'P1' by rotating D around C. This is a reliable method.
% \tkzDefPointBy[rotation=center C angle 25](D)  \tkzGetPoint{P1}

% % --- Define Line 2 (passing through B at the correct angle) ---
% % The absolute angle of the line segment BA should be 210 degrees.
% % We create a helper point 'P2' on this line in a robust two-step way:
% % 1. Define a point 'P_helper' at a 210-degree angle from the origin.
% % 2. Create P2 by moving 'P_helper' by the same vector that goes from C to B.
% \tkzDefPointBy[rotation=center B angle 30](D)
% \tkzGetPoint{P2}

% % --- Find A ---
% % A is the intersection of the line (C, P1) and the line (B, P2).
% \tkzInterLL(C,P1)(B,P2)
% \tkzGetPoint{A}

% % STEP 3: Draw the quadrilateral and its diagonals.
% \tkzDrawPolygon[thick, fill=gray!10](A,B,C,D)
% \tkzDrawSegment(A,C)
% \tkzDrawSegment(B,D)

% % STEP 4: Label the vertices.
% \tkzLabelPoint[above](A){\large{A}}
% \tkzLabelPoint[below](B){\large B}
% \tkzLabelPoint[below right](C){\large C}
% \tkzLabelPoint[right](D){\large D}

% % STEP 5: Mark the given angles and equal sides.
% \tkzMarkAngle[size=0.8, fill=blue!20,](D,B,A)
% \tkzLabelAngle[pos=-1.3](A,B,D){$30^\circ$}
% \tkzMarkAngle[size=0.8, fill=green!20](A,C,B)
% \tkzLabelAngle[pos=-1.15](B,C,A){$75^\circ$}
% \tkzMarkAngle[size=0.6, fill=red!20](D,C,A)
% \tkzLabelAngle[pos=-1.3](A,C,D){$25^\circ$}
% \tkzMarkSegment[mark=||, size=3pt](C,B)
% \tkzMarkSegment[mark=||, size=3pt](C,D)

% % STEP 6: Draw the circumcircle of triangle DAC.
% \tkzDefCircle[circum](D,A,C)
% \tkzGetPoint{O}
% \tkzDrawCircle[color=orange, thick](O,A)

% % % STEP 7: Extend line CB to find E on the circle.
% \tkzInterLC(C,B)(O,A) \tkzGetPoints{E}{C_on}
% \tkzDrawLine[color=gray, dashed](B,E)
% \tkzDrawPoint[color=red, size=5pt, fill=red](E)
% \tkzLabelPoint[above right, color=red](E){\large E}

% % % STEP 8: Highlight the segments BD and CE for the proof.
% \tkzDrawSegment[color=purple, ultra thick, opacity=0.8](B,D)
% \tkzDrawSegment[color=teal, ultra thick, opacity=0.8](C,E)

% \end{tikzpicture}

% the figure satisfies:
% \begin{eqnarray}
% 	\angle ABD &=& 30^\circ \nonumber\\
% \angle BCA &=& 75^\circ \nonumber\\
% \angle ACD &=& 25^\circ \nonumber\\
% \overline{CB} &=& \overline{CD} \nonumber
% \end{eqnarray}

% Extend the line $\overline{CB}$ beyond $B$ to meet the circumcircle of triangle ${DAC}$ again at $E$
% (so $E$ is the second intersection of line $\overline{CB}$ with the circumcircle of $\triangle DAC$).\\
% \begin{center}
% \textbf{Prove that: } $\overline{CE} = \overline{BD}$
	
% \end{center}
% \end{multicols}

% \textbf{Tiny Hint:}
% \textit{Place a point $F$ on $\overline{AB}$ (or its extension) such that $\overline{CF} = \overline{CB}$.
% Show that this creates a $60^{\circ}$ angle at $C$, then relate $\overline{CE}$ and $\overline{BD}$ using the circle.
% }


% Consider a $4 \times 4$ matrix whose entries are the integers 1 through 16 written in order left to right, top to bottom.
% Choose any set of four entries with the rule that exactly one chosen entry lies in each row and exactly one chosen entry lies in each column (equivalently: pick one entry from every row and the chosen column indices are all distinct).
% Prove that the sum of the four chosen entries is always 34. Henceforth, generalize the result for an $N \times N$ matrix.
% \\

% Clarification for students: “\textit{Exactly one chosen entry in each row and each column}” means no two chosen numbers share the same row or the same column. Such a choice is called a \textbf{transversal} (or a permutation selection) of the matrix.
% \\

% Short illustrative examples: For the $4 \times 4$ matrix,\\
% \[\begin{bmatrix}
% 1 & 2 & 3 & 4 \\
% 5 & 6 & 7 & 8 \\
% 9 & 10 & 11 & 12 \\
% 13 & 14 & 15 & 16
% \end{bmatrix}\] \\

% Main diagonal selection:  $1 + 6 + 11 + 16 = 34$.\\
% Another transversal: Choose entries at positions $(1, 4), (2, 1), (3, 2) and (4, 3)$, giving  $4 + 5 + 10 + 15 = 34$.\\
% These different choices give the same total — you must prove why this always happens.
\end{exer*}

% \vfill
% \vfill
% \begin{figure}[H]
% \centering
%     \includegraphics[width=0.3\textwidth]{./solution-qr-code.png}
%     \captionsetup{labelformat = empty}
%     \caption{\Large{(Scan this to submit solution!)}}
% \end{figure}


\bigskip
\vspace{1.6cm}


\begin{multicols}{2}
		\begin{center}
		\includegraphics[width=4.28cm, height=4.2cm]{solution-qr-code.png}	\\
		\noindent (Scan this to submit solution!) \\
			
		\end{center}
		\columnbreak
		\Large
		\textbf{\color{red}{{Top Submissions for last Week's Problem: }		}}
		
		\begin{enumerate}
			\item \textbf{Hamza Nazim Siddiqui}\\AD24B1023
			\item \textbf{Lanka Sree Lalith Karthik}\\CS23B1042
		\end{enumerate}
\end{multicols}

\pagenumbering{gobble}
\end{document}


